\documentclass[12pt, answers]{exam}
\usepackage[margin=1in, a4paper]{geometry}

\usepackage{amsmath, amsthm, amssymb}
\usepackage{siunitx}
\usepackage{tikz}
\usepackage{setspace}
\usepackage{cancel}
\usepackage{cases}
\usepackage[nolabel, final]{showlabels}

\usepackage{notomath, noto-mono}

\newcommand{\reals}{\mathbb{R}}
\newcommand{\ints}{\mathbb{Z}}
\newcommand{\posints}{\mathbb{Z}^{+}}
\newcommand{\rationals}{\mathbb{Q}}
\newcommand{\complexes}{\mathbb{C}}
\renewcommand{\frac}[2]{\dfrac{#1}{#2}}

\pagestyle{plain}

\begin{document}
\onehalfspacing%
\begin{center}
	\Large
	\textbf{Problem Of The Day 2022}
\end{center}

\begin{questions}
	\question (\textbf{21 Mar}) Simplify the algebraic fraction
	\( \frac{a^4-a^2b^2}{(a-b)^2} \div \frac{a(a+b)}{b^2} \times \frac{b}{a} \).

	\begin{solution}
		\begin{align*}
			 & \frac{a^4-a^2b^2}{(a-b)^2} \div \frac{a(a+b)}{b^2} \times \frac{b^2}{a}                                              \\
			 & = \frac{\cancel{a^2(a+b)(a-b)}}{(a-b)^{\cancel{2}}} \times \frac{b^2}{\cancel{a(a+b)}} \times \frac{b^2}{\cancel{a}} \\
			 & = \frac{b^4}{a-b}
		\end{align*}
	\end{solution}

	\question (\textbf{22 Mar}) Factorise \(a^4+a^2b^2+b^2\).
	\begin{solution}
		\begin{align*}
			a^4 + a^2b^2 + b^4 & = a^4 + 2a^2b^2 + b^4 - a^2b^2     \\
			                   & = (a^2 + b^2)^2 - (ab)^2           \\
			                   & = (a^2 - ab + b^2)(a^2 + ab + b^2)
		\end{align*}
	\end{solution}

	\question (\textbf{23 Mar}) Simplify \(\frac{1}{a-x}-\frac{1}{a+x}-\frac{2x}{a^2+x^2}-\frac{4x^3}{a^4+x^4}+\frac{8x^7}{a^8-x^8}\).
	\begin{solution}
		\begin{align*}
			 & \frac{1}{a-x}-\frac{1}{a+x}-\frac{2x}{a^2+x^2}-\frac{4x^3}{a^4+x^4}+\frac{8x^7}{a^8-x^8} \\
			 & = \frac{2x}{a^2-x^2}-\frac{2x}{a^2+x^2}-\frac{4x^3}{a^4+x^4}+\frac{8x^7}{a^8-x^8}        \\
			 & = \frac{4x^3}{a^4-x^4}-\frac{4x^3}{a^4+x^4}+\frac{8x^7}{a^8-x^8}                         \\
			 & = \frac{8x^7}{a^8-x^8}+\frac{8x^7}{a^8-x^8}                                              \\
			 & = \frac{16x^7}{a^8-x^8}
		\end{align*}
	\end{solution}

	\question (\textbf{24 Mar}) Factorise completely
	\(64x^6 - y^{12}\).
	\begin{solution}
		\begin{align*}
			64x^6 - y^{12} & = (8x^3 + y^6)(8x^3 - y^6)                                     \\
			               & = (2x + y^2)(2x - y^2)(4x^2 + 2xy^2 + y^4)(4x^2 - 2xy^2 + y^4) \\
		\end{align*}
	\end{solution}

	\question (\textbf{25 Mar}) Factorise completely \(x^2(x-1)^2 + 32(x-x^2) + 60\).
	\begin{solution}
		\begin{align*}
			 & x^2(x-1)^2 + 32(x-x^2) + 60             \\
			 & = x^2(x-1)^2 - 32x(x-1) + 60            \\
			 & = [x(x-1)]^2 - 32[x(x-1)] + 16^2 - 14^2 \\
			 & = [x(x-1) - 16]^2 - 14^2                \\
			 & = (x^2-x-2)(x^2-x-30)                   \\
			 & = (x-6)(x-2)(x+1)(x+5)
		\end{align*}
	\end{solution}

	\question (\textbf{28 Mar}) Simplify
	\(\frac{x^2-4}{x^2-4x+4}+\frac{2-x}{x+2}\).
	\begin{solution}
		\begin{align*}
			 & \frac{x^2-4}{x^2-4x+4} + \frac{2-x}{x+2}                              \\
			 & = \frac{(x+2)\cancel{(x-2)}}{(x-2)^{\cancel{2}}} + \frac{-(x-2)}{x+2} \\
			 & = \frac{(x+2)^2 - (x-2)^2}{(x+2)(x-2)}                                \\
			 & = \frac{8x}{x^2 - 4}
		\end{align*}
	\end{solution}

	\question (\textbf{29 Mar}) An equation in \(x\),
	\(\frac{m}{x-1} + \frac{3}{1-x} = 1\), has a positive solution.
	Find the possible range of values for \(m\).
	\begin{solution}
		\begin{align*}
			\frac{m}{x-1} + \frac{3}{1-x} & = 1     \\
			m - 3                         & = x - 1 \\
			x                             & = m - 2 \\
			\because x                    & > 0     \\
			\therefore m                  & > 2     \\
		\end{align*}
	\end{solution}

	\question (\textbf{30 Mar}) Given that \(\frac{1}{x} + \frac{1}{y} = 3\),
	find the value of \(\frac{3x + 4xy + 3y}{x + 2xy + y}\).
	\begin{solution}
		\begin{align*}
			\because \frac{1}{x} + \frac{1}{y} & = 3                                                            \\
			\therefore x + y                   & = 3xy                                                          \\
			\\
			\frac{3x+4xy+3y}{x+2xy+y}          & = \frac{\frac{13}{3}\cancel{(x+y)}}{\frac{5}{3}\cancel{(x+y)}} \\
			                                   & = \frac{13}{5}                                                 \\
		\end{align*}
	\end{solution}

	\question (\textbf{31 Mar})
	A factory scheduled to manufacture 480 toys within a number of days.
	The factory increased its daily production by 50\% since the beginning
	of the production and finished the whole batch of 480 toys 10 days earlier
	than the original schedule. How many toys per day did the factory plan
	to manufacture originally?

	\begin{solution}
		Let \(x\) be the number of toys per day the factory planned to manufacture originally.
		\begin{align*}
			\frac{480}{x} - \frac{480}{\frac{3}{2}x} & = 10    \\
			\frac{720x - 480x}{\frac{3}{2}x^2}       & = 10    \\
			240x                                     & = 15x^2 \\
			x                                        & = 16
		\end{align*}
		The company originally planned to manufacture 16 toys per day.
	\end{solution}

	\question (\textbf{1 Apr})
	Make \(y\) the subject of the formula:
	\[\sqrt{\frac{x^3-x+y}{xy}} = x\]
	\begin{solution}
		\begin{align*}
			\sqrt{\frac{x^3-x+y}{xy}} & = x                                                    \\
			x^2                       & = \frac{x^3-x+y}{xy}                                   \\
			x^3y                      & = x^3-x+y                                              \\
			y(x^{3}-1)                & = x^{3}-x                                              \\
			y                         & = \frac{x^{3}-x}{x^{3}-1}                              \\
			                          & = \frac{x(x+1)\cancel{(x-1)}}{\cancel{(x-1)}(x^2+x+1)} \\
			                          & = \frac{x(x+1)}{x^2+x+1}
		\end{align*}
	\end{solution}

	\question (\textbf{4 Apr}) Solve the SLEs:
	\begin{numcases}{}
		\frac{1}{x} + \frac{1}{y} & = 5     \\
		xy                          & = \(x - y\)
	\end{numcases}
	\begin{solution}
		\begin{align*}
			\text{From (1):}                                     &                                                                                               \\
			xy                                                   & = \frac{x+y}{5} \text{\quad(3)}                                                               \\
			\text{Substitute (3) into (2):}                      &                                                                                               \\
			x+y                                                  & = 5x - 5y                                                                                     \\
			4x                                                   & = 6y                                                                                          \\
			x                                                    & = \frac{3}{2}y \text{\quad(4)}                                                                \\
			\text{Substitute (4) into (2):}                      &                                                                                               \\
			\frac{3}{2}y^{2}                                     & = \frac{1}{2}y                                                                                \\
			y\left(\frac{3}{2}y-\frac{1}{2}\right)               & = 0                                                                                           \\
			\therefore y                                         & = 0 \left(\text{rej., } \frac{1}{y} \text{ cannot be undefined}\right) \text{or } \frac{1}{3} \\
			\text{Substitute } y = \frac{1}{3} \text{ into (2):} &                                                                                               \\
			\frac{1}{3}x                                         & = x - \frac{1}{3}                                                                             \\
			\frac{2}{3}x                                         & = \frac{1}{3}                                                                                 \\
			x                                                    & = \frac{1}{2}                                                                                 \\
			\therefore x                                         & = \frac{1}{2} \text{ and } y = \frac{1}{3}
		\end{align*}
	\end{solution}

	\question (\textbf{5 Apr})
	\begin{parts}
		\part Find the equation of the line \(l_{1}\) that makes a \ang{45} angle with the positive \(x\)-axis and its \(y\)-intercept is \(-3\).
		\begin{solution}
			Since the angle between the positive \(x\)-axis and \(l_{1}\) has to be \ang{45}, the gradient (\(m\)) can only be \(1\).
			Since the \(y\)-intercept is \(-3\), the value of \(c\) in the gradient-intercept form has to be \(-3\) as well.
			Therefore, the equation of line \(l_{1}\) is \(y = x - 3\).
		\end{solution}

		\part Hence, find the equation of a vertical line \(l_{2}\) which
		intersects with \(l_{1}\) at a point with a \(y\)-coordinate of 10.
		\begin{solution}
			\begin{align*}
				y            & = x - 3 \\
				10           & = x - 3 \\
				\therefore x & = 13
			\end{align*}
			The equation of line \(l_{2}\) is \(x = 13\).
		\end{solution}

		\part Are the following three points collinear:
		the intersection point between \(l_{1}\) and \(l_{2}\),
		the origin, and \((-6.5, -5)\)?
		\begin{solution}
			There can be a line, \(l_{3}\), with both the origin and the intersection on it.
			The equation of this line would thus be:
			\begin{equation*}
				y = \frac{10 - 0}{13 - 0}x = \frac{10}{13}x
			\end{equation*}
			Substituting \(x = -6.5\) and \(y = -5\) into that equation,
			we see that:
			\[
				-5 = \frac{10}{13} \times -\frac{13}{2}
			\]
			Therefore, all three points are collinear.
		\end{solution}
	\end{parts}

	\question (\textbf{7 Apr}) Expand and simplify the expression \(\left[x\left(x^{4}-y^{4}\right)-3xy\left(-y\right)^{3}\right] \cdot x^{2}y\).
	\begin{solution}
		\begin{align*}
			 & \left[x\left(x^{4}-y^{4}\right)-3xy\left(-y\right)^{3}\right] \cdot x^{2}y \\
			 & = x^{3}y\left[x^{4}-y^{4}-3y\left(-y\right)^{3}\right]                     \\
			 & = x^{3}y\left(x^{4}+2y^{4}\right)
		\end{align*}
	\end{solution}

	\question (\textbf{8 Apr}) There is a rectangular piece of paper
	with length \((5a + 4b)\) and breadth \((4a + 3b)\). A square of
	side length \((a + b)\) is cut away from the four corners of the
	paper. Using the remaining paper, a box without its top cover
	can be folded.
	\begin{parts}
		\part Find the total surface area of the folded paper box.
		\begin{solution}
			Draw the diagram.

			\begin{tikzpicture}
				\draw [red, dashed] (0, 0) -- node[below] {\(a + b\)} (2, 0);
				\draw [red, dashed] (0, 0) -- (0, 2);
				\draw (0, 2) -- (2, 2);
				\draw (2, 2) -- (2, 0);
				\draw (0, 2) -- (0, 6);
				\draw (0, 6) -- (2, 6);
				\draw [blue, dashed] (2, 6) -- node[right] {\(2a + b\)} (2, 2);
				\draw [red, dashed] (0, 6) -- (0, 8);
				\draw [red, dashed] (0, 8) -- (2, 8);
				\draw (2, 8) -- (2, 6);
				\draw (2, 8) -- node[above] {\(5a + 4b\)} (8, 8);
				\draw [latex-](0, 8.25) -- (4.25, 8.25);
				\draw [-latex](5.75, 8.25) -- (10, 8.25);
				\draw (8, 8) -- (8, 6);
				\draw [blue, dashed] (8, 6) -- (8, 2);
				\draw (8, 6) -- (10, 6);
				\draw (10, 6) -- node[right] {\(4a + 3b\)}( 10, 2);
				\draw [-latex](10.25, 4.25) -- (10.25, 8);
				\draw [-latex](10.25, 3.75) -- (10.25, 0);
				\draw (10, 2) -- (8, 2);
				\draw (8, 2) -- (8, 0);
				\draw (8, 0) -- (2, 0);
				\draw [magenta, dashed] (8, 2) -- node[above] {\(3a + 2b\)}(2, 2);
				\draw [magenta, dashed] (8, 6) -- (2, 6);
				\draw [red, dashed] (8, 8) -- (10, 8);
				\draw [red, dashed] (10, 8) -- (10, 6);
				\draw [red, dashed] (10, 2) -- (10, 0);
				\draw [red, dashed] (10, 0) -- (8, 0);
			\end{tikzpicture}

			\begin{align*}
				\text{total surface area}
				 & = 2 \cdot \left[\left(5a + 4b\right)\left(4a+3b\right) - 4(a+b)\right] \\
				 & = 2\left(20a^{2}+15ab+16ab+12b^{2}-4\left(a+b\right)\right)            \\
				 & = 40a^{2} + 62ab + 24b^{2} - 8a - 8b
			\end{align*}
		\end{solution}
		\part Find the volume of the paper box.
		\begin{solution}
			\begin{align*}
				\text{volume of paper box} & = (3a+2b)(2a+b)(a+b)                   \\
				                           & = 6a^{3} + 13a^{2}b + 9ab^{2} + 2b^{3}
			\end{align*}
		\end{solution}
	\end{parts}

	\question (\textbf{11 Apr}) Solve the equation
	\((x+1)(1-3x)(4y-5) = 7(5-4y)\).
	\begin{solution}
		\\
		\textbf{Case 1}: Solve for \(x\).
		\begin{align*}
			(x+1)(1-3x)\cancel{(4y-5)}      & = -7\cancel{(4y-5)} \\
			-3x^{2}	- 2x + 1 + 7             & = 0                 \\
			(-3x + 4)(x + 2)                & = 0                 \\
			\therefore x = -2 \text{ or } x & = \frac{4}{3}
		\end{align*}

		Both possibilities of \(x\) in \textbf{Case 1} result
		in an identity when trying to solve for \(y\), so we can say
		when \(x = -2\) or \(\frac{4}{3}\), \(y \in \reals\).

		\textbf{Case 2}: Solve for \(y\).
		\begin{align*}
			(x+1)(1-3x)(4y-5)                        & = 7(5-4y)                                        \\
			-12x^{2}y + 15x^{2} - 8xy + 10x + 4y - 5 & = -28y + 35                                      \\
			y \left(-12x^{2} - 8x + 32 \right)       & = -15x^{2} - 10x + 40                            \\
			y                                        & = \frac{-15x^{2} - 10x + 40}{-12x^{2} - 8x + 32} \\
			                                         & = \frac{5}{4}
		\end{align*}
		The solution \(y = \frac{5}{4}\) results in an identity
		when trying to solve for \(x\), so we can also say that when
		\(y = \frac{5}{4}\), \(x \in \reals\).

		\[
			\therefore
			\begin{cases}
				x = -2 \text{ or } \frac{4}{3}, & y \in \reals    \\
				x \in \reals,                   & y = \frac{5}{4}
			\end{cases}
		\]
	\end{solution}

	\question(\textbf{12 Apr}) Out of the following equations:
	\[2x^{2} + x - 3\]
	\[\frac{5x}{x^{2} + 1} = 2\]
	\[ (x + 1)(x - 2) = x^{2} \]
	\[ \left(t + 1\right)^{2} = 2t(t + 1) \]
	How many quadratic equations are there?
	\begin{solution}
		There are \textbf{2} quadratic equations.
		\begin{enumerate}
			\item	\(2x^{2} + x - 3\) is an expression, not an equation.
			\item \(\frac{5x}{x^{2} + 1}\) can be expressed in the form \(ax^{2} + bx + c = 0\).
			      \begin{align*}
				      \frac{5x}{x^{2} + 1} & = 2                       \\
				      5x                   & = 2\left(x^{2} + 1\right) \\
				      2x^{2} - 5x + 2      & = 0
			      \end{align*}
			\item \((x + 1)(x - 2) = x^{2}\) cannot be expressed in the form \(ax^{2} + bx + c = 0\). It is in fact, a linear equation.
			      \begin{align*}
				      (x + 1)(x - 2) & = x^{2} \\
				      x^{2} - x - 2  & = x^{2} \\
				      -x - 2         & = 0
			      \end{align*}
			\item \(\left(t + 1\right)^{2} = 2t(t + 1)\) can be expressed in the form \(ax^{2} = c\).
			      \begin{align*}
				      \left(t + 1\right)^{2}  & = 2t(t + 1)            \\
				      t^{2} + \cancel{2t} + 1 & = 2t^{2} + \cancel{2t} \\
				      t^{2}                   & = 1
			      \end{align*}
		\end{enumerate}
	\end{solution}

	\question (\textbf{13 Apr})
	For an equation in \(x\), \(\left(m^{2} - 8m + 17\right)x^{2} - 2mx + 1 = 0\), where \(m\) is a real constant.
	Which of the following statements is true?

	\begin{choices}
		\choice It is a quadratic equation.
		\choice It is a linear equation.
		\choice It cannot be determined if it is a quadratic equation.
	\end{choices}

	\begin{solution}
		\textbf{C}.

		Although it is true that \(m^{2} - 8m + 17 \in \reals\) and \(-2m \in \reals\),
		which makes the equation seem to conform to the standard form of
		\(ax^{2} + bx + c = 0\), there is a possibility that
		\(m^{2} - 8m + 17 = 0\), which makes the \(ax^{2}\) term \(0\),
		causing the equation to no longer be quadratic. Due to the value of \(m\) being unknown, we cannot confirm
		whether the equation is quadratic.
	\end{solution}

	\question (\textbf{14 Apr}) Solve the quadratic equation
	\((x - 1)^{2} - 5x(x + 5) = -2(x - 2) + 3\).
	\begin{solution}
		\begin{align*}
			(x - 1)^{2} - 5x(x + 5)                          & = -2(x - 2) + 3              \\
			x^{2} - 2x + 1 - 5x^{2} - 25x                    & = -2x + 7                    \\
			-4x^{2} - 25x - 6                                & = 0                          \\
			\left(-4x^{2} - x\right) + \left(-24x - 6\right) & = 0                          \\
			(4x + 1)(- x - 6)                                & = 0                          \\
			\therefore x                                     & = -6 \text{ or} -\frac{1}{4}
		\end{align*}
	\end{solution}

	\question (\textbf{18 Apr}) A rectangle of breadth \qty{2}{\centi\metre} is cut from a square.
	The rectangle's length is the same as the square's.
	The remaining area is \qty{48}{\square\centi\metre}.
	Find the original area of the square.
	\begin{solution}
		Let the length of the square be \(x\) \si{\centi\metre}.
		\begin{align*}
			x^{2} - 2x                  & = 48 \\
			(x + 6)(x - 8)              & = 0  \\
			x = -6 \text{ (rej.) or } x & = 8  \\
			\therefore  x^{2}           & = 64
		\end{align*}
		The original area of the square is \qty{64}{\square\centi\metre}.
	\end{solution}

	\question (\textbf{19 Apr}) Determine if the following relationships are quadratic functions.
	\begin{parts}
		\part Within a certain distance, the speed of the train \(v\) \si{\kilo\metre\per\hour} and the time \(t\) \si{\hour}.
		\begin{solution}
			No. Both \(v\) and \(t\) are constants within that certain distance.
		\end{solution}

		\part The perimeter \(C\) and the length of the side \(a\) of an equilateral triangle.
		\begin{solution}
			No. \(C = 3a\).
		\end{solution}

		\part The area \(S\) and the length of radius \(R\) of a sector of angle \ang{120}.
		\begin{solution}
			Yes. \(S = \frac{1}{3} \pi r^{2}\).
		\end{solution}
	\end{parts}

	\question (\textbf{20 Apr}) There is a \qty{56}{\centi\metre} piece of wire.
	It is cut into two pieces and each piece is then made into a square.
	\begin{parts}
		\part If the sum of the areas of the two squares is \qty{100}{\square\centi\metre},
		how should the wire be cut?
		\begin{solution}
			Let the lengths of the two pieces be \(x\) \si{\centi\metre} and \(y\) \si{\centi\metre} respectively.
			\begin{align}
				x + y                                                          & = 56 \label{eq:0420a1} \tag{1}  \\
				{\left(\frac{x}{4}\right)}^{2}+ {\left(\frac{y}{4}\right)}^{2} & = 100 \label{eq:0420a2} \tag{2}
			\end{align}
			From \eqref{eq:0420a2}:
			\begin{align*}
				\frac{x^{2} + y^{2}}{16} & = 100                            \\
				x^{2} + y^{2}            & = 1600 \label{eq:0420a3} \tag{3} \\
			\end{align*}
			Substitute \eqref{eq:0420a1} into \eqref{eq:0420a3}:
			\begin{align*}
				x^{2} + (56 - x)^{2}        & = 1600                                        \\
				x^{2} + 3136 - 112x + x^{2} & = 1600                                        \\
				2x^{2}  - 112x + 1536       & = 0                                           \\
				x^{2}  - 56x + 768          & = 0                                           \\
				(x - 24)(x - 32)            & = 0                                           \\
				x                           & = 24 \text{ or } 32 \label{eq:0420a4} \tag{4} \\
			\end{align*}
			Substitute \eqref{eq:0420a4} into \eqref{eq:0420a1}:
			\[
				\begin{cases}
					x = 24, y = 32 \\
					x = 32, y = 24 \\
				\end{cases}
			\]
			Hence, the wire should be cut into \qty{24}{\centi\metre} and \qty{32}{\centi\metre} pieces.
		\end{solution}

		\part If the sum of the areas of the two squares is \qty{196}{\square\centi\metre},
		how should the wire be cut?
		\begin{solution}
			Let the lengths of the two pieces be \(x\) \si{\centi\metre} and \(y\) \si{\centi\metre} respectively.
			\begin{align}
				x + y                                                          & = 56 \label{eq:0420b1} \tag{1}  \\
				{\left(\frac{x}{4}\right)}^{2}+ {\left(\frac{y}{4}\right)}^{2} & = 196 \label{eq:0420b2} \tag{2}
			\end{align}
			From \eqref{eq:0420b2}:
			\begin{align*}
				\frac{x^{2} + y^{2}}{16} & = 196                            \\
				x^{2} + y^{2}            & = 3136 \label{eq:0420b3} \tag{3} \\
			\end{align*}
			Substitute \eqref{eq:0420b1} into \eqref{eq:0420b3}:
			\begin{align*}
				x^{2} + (56 - x)^{2}        & = 3136                    \\
				x^{2} + 3136 - 112x + x^{2} & = 3136                    \\
				2x^{2}  - 112x              & = 0                       \\
				x^{2}  - 56x                & = 0                       \\
				x( x - 56 )                 & = 0                       \\
				x                           & = 0 \text{ (rej.) or } 56
			\end{align*}
			Hence, the wire cannot be cut into two such pieces.
		\end{solution}

		\part Is it possible for the sum of the two areas to be \qty{200}{\square\centi\metre}?
			\begin{solution}
Assuming that it \textbf{is} possible:
			Let the lengths of the two pieces be \(x\) \si{\centi\metre} and \(y\) \si{\centi\metre} respectively.
			\begin{align}
				x + y                                                          & = 56 \label{eq:0420c1} \tag{1}  \\
				{\left(\frac{x}{4}\right)}^{2}+ {\left(\frac{y}{4}\right)}^{2} & = 200 \label{eq:0420c2} \tag{2}
			\end{align}
			From \eqref{eq:0420c2}:
			\begin{align*}
				\frac{x^{2} + y^{2}}{16} & = 200                            \\
				x^{2} + y^{2}            & = 3200 \label{eq:0420c3} \tag{3} \\
			\end{align*}
			Substitute \eqref{eq:0420c1} into \eqref{eq:0420c3}:
			\begin{align*}
				x^{2} + (56 - x)^{2}        & = 3200                    \\
				x^{2} + 3136 - 112x + x^{2} & = 3200                    \\
				2x^{2}  - 112x - 64             & = 0                       \\
				x^{2}  - 56x -32               & = 0                       \\
				x& = \frac{56 \pm \sqrt{3264}}{2}                       \\
			\end{align*}
			Even if we reject the negative case, \(x = \frac{56 - \sqrt{3264}}2\),
			\(y\) will still be negative,
			rendering it impossible for the sum of the areas of the two squares to be \qty{200}{\square\centi\metre}.
			\end{solution}
	\end{parts}
\end{questions}
\end{document}
