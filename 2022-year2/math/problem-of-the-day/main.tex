\documentclass[12pt, answers]{exam}

\usepackage[margin=1in, a4paper]{geometry}

% \renewcommand{\familydefault}{\sfdefault}
% \usepackage{plex-otf}
% \usepackage{eulervm}

% \usepackage[cmintegrals, cmbraces]{newtxmath}
% \usepackage{ebgaramond-maths}

\usepackage{amsmath, amssymb}
\usepackage{siunitx}
\usepackage{tikz}
\usepackage{setspace}
\usepackage{cancel}

\pagestyle{plain}

\begin{document}
\onehalfspacing%
\begin{center}
	\Large
	\textbf{Problem Of The Day 2022}
\end{center}

\begin{questions}

	%%% 1
	\question (\textbf{21 Mar}) Simplify the algebraic fraction
	\(\dfrac{a^4-a^2b^2}{(a-b)^2} \div \dfrac{a(a+b)}{b^2} \times \dfrac{b}{a}\).

	\begin{solution}
		\begin{align*}
			 & \dfrac{a^4-a^2b^2}{(a-b)^2} \div \dfrac{a(a+b)}{b^2} \times \dfrac{b^2}{a}                                            \\
			 & = \dfrac{\cancel{a^2(a+b)(a-b)}}{(a-b)^\cancel{2}} \times \dfrac{b^2}{\cancel{a(a+b)}} \times \dfrac{b^2}{\cancel{a}} \\
			 & = \dfrac{b^4}{a-b}
		\end{align*}
	\end{solution}


	%%% 2
	\question (\textbf{22 Mar}) Factorise \(a^4+a^2b^2+b^2\).
	\begin{solution}
		\begin{align*}
			a^4 + a^2b^2 + b^4 & = a^4 + 2a^2b^2 + b^4 - a^2b^2     \\
			                   & = (a^2 + b^2)^2 - (ab)^2           \\
			                   & = (a^2 - ab + b^2)(a^2 + ab + b^2)
		\end{align*}
	\end{solution}


	%%% 3
	\question (\textbf{23 Mar}) Simplify \(\dfrac{1}{a-x}-\dfrac{1}{a+x}-\dfrac{2x}{a^2+x^2}-\dfrac{4x^3}{a^4+x^4}+\dfrac{8x^7}{a^8-x^8}\).
	\begin{solution}
		\begin{align*}
			 & \dfrac{1}{a-x}-\dfrac{1}{a+x}-\dfrac{2x}{a^2+x^2}-\dfrac{4x^3}{a^4+x^4}+\dfrac{8x^7}{a^8-x^8} \\
			 & = \dfrac{2x}{a^2-x^2}-\dfrac{2x}{a^2+x^2}-\dfrac{4x^3}{a^4+x^4}+\dfrac{8x^7}{a^8-x^8}         \\
			 & = \dfrac{4x^3}{a^4-x^4}-\dfrac{4x^3}{a^4+x^4}+\dfrac{8x^7}{a^8-x^8}                           \\
			 & = \dfrac{8x^7}{a^8-x^8}+\dfrac{8x^7}{a^8-x^8}                                                 \\
			 & = \dfrac{16x^7}{a^8-x^8}
		\end{align*}
	\end{solution}

	%%% 4
	\question (\textbf{24 Mar}) Factorise completely
	\(64x^6 - y^{12}\).
	\begin{solution}
		\begin{align*}
			64x^6 - y^{12} & = (8x^3 + y^6)(8x^3 - y^6)                                     \\
			               & = (2x + y^2)(2x - y^2)(4x^2 + 2xy^2 + y^4)(4x^2 - 2xy^2 + y^4) \\
		\end{align*}
	\end{solution}

	%%% 5
	\question (\textbf{25 Mar}) Factorise completely
	\(x^2(x-1)^2 + 32(x-x^2) + 60\).
	\begin{solution}
		\begin{align*}
			 & x^2(x-1)^2 + 32(x-x^2) + 60             \\
			 & = x^2(x-1)^2 - 32x(x-1) + 60            \\
			 & = [x(x-1)]^2 - 32[x(x-1)] + 16^2 - 14^2 \\
			 & = [x(x-1) - 16]^2 - 14^2                \\
			 & = (x^2-x-2)(x^2-x-30)                   \\
			 & = (x-6)(x-2)(x+1)(x+5)
		\end{align*}
	\end{solution}

	%%% 6
	\question (\textbf{28 Mar}) Simplify
	\(\dfrac{x^2-4}{x^2-4x+4}+\dfrac{2-x}{x+2}\).
	\begin{solution}
		\begin{align*}
			 & \dfrac{x^2-4}{x^2-4x+4} + \dfrac{2-x}{x+2}                            \\
			 & = \dfrac{(x+2)\cancel{(x-2)}}{(x-2)^\cancel{2}} + \dfrac{-(x-2)}{x+2} \\
			 & = \dfrac{(x+2)^2 - (x-2)^2}{(x+2)(x-2)}                               \\
			 & = \dfrac{8x}{x^2 - 4}
		\end{align*}
	\end{solution}

	%%% 7
	\question (\textbf{29 Mar}) An equation in \(x\),
	\(\dfrac{m}{x-1} + \dfrac{3}{1-x} = 1\), has a positive solution.
	Find the possible range of values for \(m\).
	\begin{solution}
		\begin{align*}
			\dfrac{m}{x-1} + \dfrac{3}{1-x} & = 1     \\
			m - 3                           & = x - 1 \\
			x                               & = m - 2 \\
			\because x                      & > 0,    \\
			\therefore m                    & >2      \\
		\end{align*}
	\end{solution}

	%%% 8
	\question (\textbf{30 Mar}) Given that \(\dfrac{1}{x} + \dfrac{1}{y} = 3\),
	find the value of \(\dfrac{3x + 4xy + 3y}{x + 2xy + y}\).
	\begin{solution}
		\begin{align*}
			\because \dfrac{1}{x} + \dfrac{1}{y} & = 3                                                             \\
			\therefore x + y                     & = 3xy                                                           \\
			\\
			\dfrac{3x+4xy+3y}{x+2xy+y}           & = \dfrac{\frac{13}{3}\cancel{(x+y)}}{\frac{5}{3}\cancel{(x+y)}} \\
			                                     & = \dfrac{13}{5}                                                 \\
		\end{align*}
	\end{solution}

	%%% 9
	\question (\textbf{31 Mar})
	A factory scheduled to manufacture 480 toys within a number of days.
	The factory increased its daily production by 50\% since the beginning
	of the production and finished the whole batch of 480 toys 10 days earlier
	than the original schedule. How many toys per day did the factory plan
	to manufacture originally?

	\begin{solution}
		Let \(x\) be the number of toys per day the factory planned to manufacture originally.
		\begin{align*}
			\dfrac{480}{x} - \dfrac{480}{\frac{3}{2}x} & = 10    \\
			\dfrac{720x - 480x}{\frac{3}{2}x^2}        & = 10    \\
			240x                                       & = 15x^2 \\
			x                                          & = 16
		\end{align*}
		The company originally planned to manufacture 16 toys per day.
	\end{solution}
\end{questions}
\end{document}
