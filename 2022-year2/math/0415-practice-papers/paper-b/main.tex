%! TEX TS-program = xelatex

\documentclass[11pt, answers, addpoints]{exam}
\usepackage{amsmath, amsthm, amssymb}
\usepackage{siunitx}
\usepackage{tikz}
\usepackage{pgfplots}
\usepackage{setspace}
\usepackage{cancel}
\usepackage{cases}
\usepackage[nolabel, final]{showlabels}
\usepackage{color}

\usepackage{notomath, noto-mono}

\newcommand{\reals}{\mathbb{R}}
\newcommand{\ints}{\mathbb{Z}}
\newcommand{\posints}{\mathbb{Z}^{+}}
\newcommand{\rationals}{\mathbb{Q}}
\newcommand{\complexes}{\mathbb{C}}
\renewcommand{\frac}[2]{\dfrac{#1}{#2}}

\pagestyle{plain}

\begin{document}
\onehalfspacing%

\begin{center}
	\Large
	\textbf{SBGE Paper B (2022)}
\end{center}

\bracketedpoints%
\pointsinrightmargin%
\definecolor{SolutionColor}{rgb}{0.8, 0.9, 1}

\begin{questions}
	\question[3]
	The latest hand sanitiser bottle is in the shape of a cone.
	The circumference of the circular base is \((44x + 44)\) and
	the height of the cone is \(\left(3x - \frac{6}{7}\right)\). Taking
	\(\pi = \frac{22}{7}\), and the volume of the cone is
	\(\frac{1}{3}\pi r^{2}h\), expand and simplify the volume of the hand
	sanitiser.

	\begin{solution}
		\begin{align*}
			\text{radius of cone} & = \frac{44x + 44}{2 \times \frac{22}{7}}                                     \\
			                      & = 7x + 7                                                                     \\
			\text{volume of cone} & = \frac{1}{3} \times \frac{22}{7} \times (7x + 7)^2 \times \frac{21x - 6}{7} \\
			                      & = \frac{1}{3} \times 22(7x + 7)(x + 1) \times \frac{21x - 6}{7}              \\
			                      & = 22(x + 1)^2 \times (7x - 2)                                                \\
			                      & = 22\left(x^2 + 2x + 1\right)(7x - 2)                                        \\
			                      & = 22\left[7x^3 + 14x^2 + 7x - \left(2x^2 + 4x + 2\right)\right]              \\
			                      & = 22\left(7x^3 + 12x^2 + 3x - 2\right)                                       \\
			                      & = 154x^3 + 264x^2 + 66x - 44
		\end{align*}
	\end{solution}

	\question Factorise the following \textbf{completely}:
	\begin{parts}
		\part[2] \(27p^{3} - 36p^{2} + 12p\)
		\begin{solution}
			\begin{align*}
				27p^{3} - 36p^{2} + 12p & = 3p\left(9p^{2} - 12p + 4\right) \\
				                        & = 3p\left(3p - 2\right)^{2}
			\end{align*}
		\end{solution}
		\part[3] \(de^{2} - e^{2}f - 4df^{2} + 4f^{3}\)
		\begin{solution}
			\begin{align*}
				de^{2} - e^{2}f - 4df^{2} + 4f^{3} & = e^{2}(d - f) - 4f^{2}(d - f)       \\
				                                   & = (d - f)\left(e^{2} - 4f^{2}\right) \\
				                                   & = (d - f)(e + 2f)(e - 2f)
			\end{align*}
		\end{solution}
	\end{parts}

	\question Simplify the following algebraic expressions.
	\begin{parts}
		\part[3] \(\frac{3y}{y^{2} - 1} + \frac{3}{1 - y}\)
		\begin{solution}
			\begin{align*}
				\frac{3y}{y^{2} - 1} + \frac{3}{1 - y} & = \frac{3y - 3(y + 1)}{y^{2} - 1} \\
				                                       & = -\frac{3}{y^{2} - 1}
			\end{align*}
		\end{solution}

		\part[4] \(\frac{a^{2} - 3ab + 2b^{2}}{(a - b)^{2}} \div \frac{2a^{2} - ab - 6b^{2}}{a^{2} - b^{2}}\)
		\begin{solution}
			\begin{align*}
				\frac{a^{2} - 3ab + 2b^{2}}{(a - b)^{2}} \div \frac{2a^{2} - ab - 6b^{2}}{a^{2} - b^{2}} & = \frac{\cancel{(a - b)}\cancel{(a - 2b)}}{\cancel{(a - b)}\cancel{^{2}}} \cdot \frac{(a + b)\cancel{(a - b)}}{\cancel{(a - 2b)}(2a + 3b)} \\
				                                                                                         & = \frac{a + b}{2a + 3b}
			\end{align*}
		\end{solution}
	\end{parts}

	\question[4] Make \(h\) the subject of the formula: \(\sqrt{\frac{h^{3}mp}{h^{3} + p}} = mp\).
	\begin{solution}
		\begin{align*}
			\sqrt{\frac{h^{3}mp}{h^{3} + p}} & = mp                                                                               \\
			\frac{h^{3}mp}{h^{3} + p}        & = m^{2}p^{2}                                                                       \\
			h^{3}mp                          & = h^{3}m^{2}p^{2} + m^{2}p^{3}                                                     \\
			h^{3}mp - h^{3}m^{2}p^{2}        & = m^{2}p^{3}                                                                       \\
			h^{3}(mp - m^{2}p^{2})           & = m^{2}p^{3}                                                                       \\
			h^{3}                            & = \frac{m\cancel{^{2}}p\cancel{^{3}}^{2}}{\cancel{m}\cancel{p}\left(1 - mp\right)} \\
			h                                & = \sqrt[3]{\frac{mp^{2}}{1 - mp}}
		\end{align*}
	\end{solution}

	\question%
	\begin{parts}
		\part[2] Solve the equation: \(\frac{3x}{x + 1} - \frac{2x}{x - 1} = 1\).
		\begin{solution}
			\begin{align*}
				\frac{3x}{x + 1} - \frac{2x}{x - 1} & = 1           \\
				3x(x - 1) - 2x(x + 1)               & = x^{2} - 1   \\
				x^{2} - 5x                          & = x^{2} - 1   \\
				-5x                                 & = -1          \\
				x                                   & = \frac{1}{5}
			\end{align*}
		\end{solution}

		\part[1] Hence or otherwise, solve the equation: \(\frac{3x - 3}{x} + \frac{2x - 2}{x - 2} = 1\).
		\begin{solution}
			\begin{align*}
				\frac{3x - 3}{x} + \frac{2x - 2}{x - 2}                 & = 1           \\
				\frac{3x - 3}{(x - 1) + 1} + \frac{2x - 2}{(x - 1) - 1} & = 1           \\
				x - 1                                                   & = \frac{1}{5} \\
				x                                                       & = \frac{6}{5}
			\end{align*}
		\end{solution}
	\end{parts}

	\question Given that \(a^2 - 121 = 9879\),
	\begin{parts}
		\part[1] Find the positive value of \(a\).
		\begin{solution}
			\begin{align*}
				a^2 - 121 & = 9879  \\
				a^{2}     & = 10000 \\
				a         & = 100
			\end{align*}
		\end{solution}

		\part[3] Hence, find two factors of \(9879\) which are between \(50\) and \(200\).
		\begin{solution}
			\begin{align*}
				a^2 - 121        & = 9879 \\
				(a + 11)(a - 11) & = 9879 \\
			\end{align*}
			Taking \(a = 100\), the factors of \(9879\) are \(100 + 11 = 111\) and \(100 - 11 = 89\).
		\end{solution}
	\end{parts}

	\question[4]
	\underline{Solve the entirety of this question using Simultaneous Linear Equations.}

	In every school, a space is required to be set aside for students who
	may exhibit any symptoms of cough or cold. In one particular school, this
	space is in the form of a rectangle of the following dimensions
	(in \si{\metre}). Find the length of the rectangle.
	\begin{figure}[htpb]
		\centering
		\begin{tikzpicture}
			\draw (0, 0) -- node[below] {\(\frac{6y + 21}{10}\)} (6, 0) -- node[right] {\(\frac{12x - 6y - 49}{12}\)} (6, 3) -- node[above] {\(\frac{2x - 1}{3}\)} (0, 3) -- node[left] {\(\frac{y - 1}{3}\)} cycle;
		\end{tikzpicture}
		\label{fig:rect}
	\end{figure}
	\begin{solution}
		\begin{align}
			\frac{2x - 1}{3} & = \frac{6y + 21}{10}\label{eq:71}\tag{1}       \\
			\frac{y - 1}{3}  & = \frac{12x - 6y - 49}{12}\label{eq:72}\tag{2}
		\end{align}
		Cross-multiply \eqref{eq:71}:
		\begin{align*}
			\frac{2x - 1}{3} & = \frac{6y + 21}{10}                       \\
			10(2x - 1)       & = 3(6y + 21)                               \\
			20x - 10         & = 18y + 63                                 \\
			20x - 73         & = 18y                                      \\
			y                & = \dfrac{20x - 73}{18}\label{eq:73}\tag{3}
		\end{align*}
		Cross-multiply \eqref{eq:72}:
		\begin{align*}
			\frac{y - 1}{3} & = \frac{12x - 6y - 49}{12} \\
			12(y - 1)       & = 3(12x - 6y - 49)         \\
			4y - 4          & = 12x - 6y - 49            \\
			10y + 45        & = 12x \label{eq:74}\tag{4}
		\end{align*}
		Substitute \eqref{eq:73} into \eqref{eq:74}:
		\begin{align*}
			10\left(\frac{20x - 73}{18}\right) + 45 & = 12x                                  \\
			100x - 365 + 405                        & = 108x                                 \\
			\therefore x                            & = \frac{405 - 365}{8}                  \\
			                                        & = 5                                    \\
			\therefore y                            & = \frac{20x - 73}{18}                  \\
			                                        & = \frac{20 \times 5 - 73}{18}          \\
			                                        & = \frac{3}{2}                          \\
			\therefore \frac{2x - 1}{3}             & = \frac{6 \times \frac{3}{2} + 21}{10} \\
			                                        & = 3 \text{ units}
		\end{align*}
		The length of the rectangle is \textbf{3 units}.
	\end{solution}

	\question The diagram below shows the line \(l_{1}\), \(y = ax + b\).
	\begin{figure}[htpb]
		\centering
		\begin{tikzpicture}
			\begin{axis}[axis lines = left, xlabel = \(x\), ylabel = \(y\), domain=0:6]
				\addplot[color = black]{6 - 0.5 * x};
				\draw[dashed] (4, 4) -- (4, 0);
				\draw[dashed] (4, 4) -- (0, 4);
			\end{axis}
		\end{tikzpicture}
		\label{fig:l1}
	\end{figure}

	\begin{parts}
		\part[2] State the values of \(a\) and \(b\).
		\begin{solution}
			\(a = -\frac{1}{2}\), \(b = 6\).
		\end{solution}

		\part[2] Find the equation of another line, \(l_{2}\), which is parallel
		to \(l_{1}\) and passes through the point \((2, 3)\).
		\begin{solution}
			On \(l_{2}\),
			\begin{align*}
				y_{2} - y_{1} & = m(x_{2} - x_{1})          \\
				y_{2} - 3     & = -\frac{1}{2}(x_{2} - 2)   \\
				y_{2}         & = -\frac{1}{2}x_{2} + 1 + 3
			\end{align*}
			The equation of \(l_{2}\) is \(y = -\frac{1}{2}x + 4\).
		\end{solution}
	\end{parts}

	\question \underline{\textbf{Attempt the whole of this question on the graph paper provided.}}

	The variables \(x\) and \(y\) are connected by the equation \(y + 2 = 3x\).

	\begin{parts}
		\part[1] Copy and complete the following table.
		\begin{tabular}{|c|c|c|c|}
			\hline
			\(x\) & \(-1\) & \(0\) & \(2\) \\
			\hline
			\(y\) &        &       &       \\
			\hline
		\end{tabular}

		\begin{solution}
			\begin{tabular}{|c|c|c|c|}
				\hline
				\(x\) & \(-1\) & \(0\)  & \(2\) \\
				\hline
				\(y\) & \(-5\) & \(-2\) & \(4\) \\
				\hline
			\end{tabular}
		\end{solution}

		\part[2] \label{part:9b} Using a scale of 4 \si{\centi\metre} to represent
		1 unit on the \(x\)-axis and a scale of 2 \si{\centi\metre} to
		represent 1 unit on the \(y\)-axis, draw the graph of \(y + 2 = 3x\)
		for \(-1 \le x \le 2\).
		\begin{figure}[htpb]
			\centering
			\begin{tikzpicture}
				\begin{axis}[domain=-1:2, axis x line=middle, axis y line=middle]
					\addplot[red, thick, mark=x] coordinates {(-1, -5) (0, -2) (2, 4)};
				\end{axis}
			\end{tikzpicture}
			\label{fig:9b}
		\end{figure}

		\part[1] \textbf{Read from your graph} the value of \(x\) when \(y = 2.3\).
		\begin{solution}
			1.4 \textit{(from graph)}. \textit{[Actual: \(1\frac{13}{30}\)]}
		\end{solution}

		\part[1] \label{part:9d} On the same axes as in (\ref{part:9b}), draw the graph of \(x = 0.5\).
		\begin{figure}[htpb]
			\centering
			\begin{tikzpicture}
				\begin{axis}[domain=-1:2, axis x line=middle, axis y line=middle]
					\addplot[red, thick, mark=x] coordinates {(-1, -5) (0, -2) (2, 4)};
					\addplot[blue, thick] {.5};
				\end{axis}
			\end{tikzpicture}
			\label{fig:9b}
		\end{figure}

		\part[1] Given the graphs you have drawn in (\ref{part:9b}) and (\ref{part:9d}), explain how to find the solution to the simultaneous equations \(y + 2 = 3x\) and \(x = 0.5\).
		\begin{solution}
			Find the coordinates of the point of intersection between the two lines. The \(x\)- and \(y\)-coordinates of the point will correspond with the solutions to \(x\) and \(y\) in the simultaneous equations.
		\end{solution}
	\end{parts}
\end{questions}

\end{document}
